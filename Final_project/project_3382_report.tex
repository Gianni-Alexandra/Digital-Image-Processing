% !TeX spellcheck = el_EL-ModernGreek
\documentclass[a4paper,12pt]{article} % This defines the style of your paper
\usepackage[left=2cm, right=2cm, top=1.50cm, bottom=1.50cm, includeheadfoot]{geometry} 

\usepackage[greek]{babel}
%\usepackage[T1]{fontenc}
\usepackage{fontspec}
%\setmainfont[Mapping=tex-text]{Linux Libertine O} % For Linux
\setmainfont[Mapping=tex-text]{Times New Roman} % For Windows

% The following two packages - multirow and booktabs - are needed to create nice looking tables.
\usepackage{tabularray}

% As we usually want to include some plots (.pdf files) we need a package for that.
\usepackage{graphicx} 

% The default setting of LaTeX is to indent new paragraphs. This is useful for articles. But not really nice for homework problem sets. The following command sets the indent to 0.
\usepackage{setspace}
\setlength{\parindent}{0in}

% Package to place figures where you want them.
\usepackage{float}

% The fancyhdr package let's us create nice headers.
\usepackage{fancyhdr}

\usepackage[pdfauthor={Alexandra Gianni},
pdftitle={Project Report},
pdfcreator={TeX},
pdfsubject={Digital Image Processing}]{hyperref}

\usepackage{wrapfig}
%\usepackage{subfigure}
\usepackage{caption}
\usepackage[labelformat=simple]{subcaption}
\renewcommand{\thesubfigure}{\textbf{(\roman{subfigure})}}
\usepackage{svg}
\usepackage{siunitx}
\usepackage{amsmath}
\usepackage{mathtools}

\hypersetup{
	colorlinks,
	citecolor=blue,
	filecolor=black,
	linkcolor=black,
	urlcolor=blue
}

%\renewcommand{\thesubfigure}{\alph{subfigure}}

%%%%%%%%%%%%%%%%%%%%%%%%%%%%%%%%%%%%%%%%%%%%%%%%
% 3. Header (and Footer)
%%%%%%%%%%%%%%%%%%%%%%%%%%%%%%%%%%%%%%%%%%%%%%%%

% To make our document nice we want a header and number the pages in the footer.

\newcommand{\Title}{Project Report}
\newcommand{\Author}{Alexandra Gianni}


\pagestyle{fancy} % With this command we can customize the header style.

\fancyhf{} % This makes sure we do not have other information in our header or footer.

\fancyhead[l]{\footnotesize \Title}
%\lhead{\footnotesize Lab1 Report}% \lhead puts text in the top left corner. \footnotesize sets our font to a smaller size.

%\rhead works just like \lhead (you can also use \chead)
\fancyhead[r]{\footnotesize Digital Image Processing}
%\rhead{\footnotesize ECE327 - Ψηφιακά Συστήματα VLSI} %<---- Fill in your lastnames.

% Similar commands work for the footer (\lfoot, \cfoot and \rfoot).
% We want to put our page number in the center.
\fancyfoot[r]{\footnotesize \thepage}
\renewcommand{\footrulewidth}{0.4pt} % Line at the footer visible
%\rfoot{\footnotesize \thepage}

\fancypagestyle{first_page}{
	\fancyhf{}
	\fancyfoot[R]{\thepage}
	\renewcommand{\headrulewidth}{0pt}
	\renewcommand{\footrulewidth}{0.4pt}
	\vspace{-1cm}
}

\newcommand{\MathSpace}{\ }

\UseTblrLibrary{booktabs}

\setcounter{secnumdepth}{0}
%%%%%%%%%%%%%%%%%%%%%%%%%%%%%%%%%%%%%%%%%%%%%%%%
% 4. Your document
%%%%%%%%%%%%%%%%%%%%%%%%%%%%%%%%%%%%%%%%%%%%%%%%

% Now, you need to tell LaTeX where your document starts. We do this with the \begin{document} command.
	% Like brackets every \begin{} command needs a corresponding \end{} command. We come back to this later.
	
	\begin{document}
		
		
		%%%%%%%%%%%%%%%%%%%%%%%%%%%%%%%%%%%%%%%%%%%%%%%%
		%%%%%%%%%%%%%%%%%%%%%%%%%%%%%%%%%%%%%%%%%%%%%%%%
		
		%%%%%%%%%%%%%%%%%%%%%%%%%%%%%%%%%%%%%%%%%%%%%%%%
		% Title section of the document
		%%%%%%%%%%%%%%%%%%%%%%%%%%%%%%%%%%%%%%%%%%%%%%%%
		
		% For the title section we want to reproduce the title section of the Problem Set and add your names.
		
		\thispagestyle{first_page} % This command disables the header on the first page. 
		
		\begin{tabular}{lr}
			& \hspace{5.2cm}\raisebox{-\totalheight}{\includegraphics[scale=0.25]{uth-logo.png}} \\ [-15mm]
			\large \bf Digital Image Processing&  \\
			University of Thessaly & \\
			Department of Electrical and Computer Engineering & \\
			%	\today & \\
		\end{tabular}
		
		\par\noindent\rule{\textwidth}{0.5pt}
		
		\vspace*{0.3cm} % Now we want to add some vertical space in between the line and our title.
		
		\begin{center} % Everything within the center environment is centered.
			{\Large \bf \Title} % <---- Don't forget to put in the right number
			\vspace{2mm}
			
			% YOUR NAMES GO HERE
			{\bf\Author \\ STUDENT ID: 3382} % <---- Fill in your names here!
			
		\end{center}  
		
		\vspace{0.4cm}
		
		% Grafw apo edw kai katw me inputs
		% !TeX spellcheck = en_US
\section{Description}
The aim of this project is to develop a Python-based script that enables users to upload and edit portrait images. This application will facilitate the creation of personalized memory photo albums, offering various editing options and filters.\\

More specifically, the script will support three primary image formats: PNG, JPEG, and TIFF. If the user wont upload an image with the supported format, then he will get notified about it and he will need to adjust to the requirements.\\
Once the user uploads an image in one of the supported formats, the application provides a versatile editing interface. The user can select from a suite of filters and feature placements in order to edit the images. Users can review their edits in real-time, make adjustments as needed, and save the finished product in JPEG format, or choose to reset the canvas to start a new project, thereby facilitating a seamless workflow for creating memorable photo albums.\\

I will break down the steps we had to follow to complete the project.\\
\begin{itemize}
	\item Step 1: Load image from requested formats or inform user about unsupported format.\\
	
	\item Step 2: Create 4 filters\\
	
	\item Step 3: Create a face and eye detection function to place elements (eg hats, glasses) on the faces and eyes. We will need to prepare a set of '.png' such elements.\\
	
	\item Step 4: Create a line detect function and a circles detect function to locate geometries, for placing frame and sticker elements. We will need to prepare a set of '.png' such elements.
	
	
	
\end{itemize}


		%\input{Sources/askhsh2.tex}
		
	\end{document}